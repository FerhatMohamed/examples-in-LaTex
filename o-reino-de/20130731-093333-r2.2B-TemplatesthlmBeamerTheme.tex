\documentclass[compress]{beamer}
\usetheme{sthlm}

%-=-=-=-=-=-=-=-=-=-=-=-=-=-=-=-=-=-=-=-=-=-=-=-=
%        LOADING BEAMER PACKAGES
%-=-=-=-=-=-=-=-=-=-=-=-=-=-=-=-=-=-=-=-=-=-=-=-=

\usepackage{
booktabs,
datetime,
dtk-logos,
graphicx,
multicol,
pgfplots,
ragged2e,
tabularx,
tikz,
wasysym
}

\pgfplotsset{compat=1.8}

\usepackage[utf8]{inputenc}
\usepackage[portuguese]{babel}
\usepackage[T1]{fontenc}
\usepackage{newpxtext,newpxmath}
\usepackage{listings}

\lstset{ %
language=[LaTeX]TeX,
basicstyle=\normalsize\ttfamily,
keywordstyle=,
numbers=left,
numberstyle=\tiny\ttfamily,
stepnumber=1,
showspaces=false,
showstringspaces=false,
showtabs=false,
breaklines=true,
frame=tb,
framerule=0.5pt,
tabsize=4,
framexleftmargin=0.5em,
framexrightmargin=0.5em,
xleftmargin=0.5em,
xrightmargin=0.5em
}



%-=-=-=-=-=-=-=-=-=-=-=-=-=-=-=-=-=-=-=-=-=-=-=-=
%        LOADING TIKZ LIBRARIES
%-=-=-=-=-=-=-=-=-=-=-=-=-=-=-=-=-=-=-=-=-=-=-=-=

\usetikzlibrary{
backgrounds,
mindmap
}

%-=-=-=-=-=-=-=-=-=-=-=-=-=-=-=-=-=-=-=-=-=-=-=-=
%        BEAMER OPTIONS
%-=-=-=-=-=-=-=-=-=-=-=-=-=-=-=-=-=-=-=-=-=-=-=-=

\setbeameroption{show notes}

%-=-=-=-=-=-=-=-=-=-=-=-=-=-=-=-=-=-=-=-=-=-=-=-=
%        BEAMER COMMANDS
%-=-=-=-=-=-=-=-=-=-=-=-=-=-=-=-=-=-=-=-=-=-=-=-=


%-=-=-=-=-=-=-=-=-=-=-=-=-=-=-=-=-=-=-=-=-=-=-=-=
%
%	PRESENTATION INFORMATION
%
%-=-=-=-=-=-=-=-=-=-=-=-=-=-=-=-=-=-=-=-=-=-=-=-=

\title{O Reino de Deus e os Últimos do Mundo}
\subtitle{Série - Ensinos do Reino - Lição 03 \\ Atualizado em: }
%\date{\small{\jobname}}
\author{\texttt{Egmon Pereira}}
\institute{\texttt{Baseado nos Estudos Ensinos do Reino - Ed. Didaquê}}

\hypersetup{
pdfauthor = {Egmon Pereira: @Egmon},      
pdfsubject = {Teologia, },
pdfkeywords = {},  
pdfmoddate= {D:\pdfdate},          
pdfcreator = {WriteLaTeX}
}

\begin{document}

\begin{frame}
\titlepage

\end{frame}

\begin{frame}{Introdução}
\centering
{\Huge Imagine que você se deparasse com Deus na porta do céu. Então ele perguntaria a você: }
\end{frame}

\begin{frame}{Introdução}
\centering
{\Huge Por que deveria eu permitir que você entre no meu céu? 
\pause
O que você responderia? }
\end{frame}

\begin{frame}{Textos Bíblicos}

\begin{itemize}

{\Huge \item[•] Mateus 20.1-16\\
Parábola da Vinha}

\end{itemize}
\end{frame}

\begin{frame}{O contexto da Parábola}
\begin{itemize}
{\huge \item[•] Mateus 19
\item[•] Encontro com o Jovem Rico.
\pause
\item[•] Os trabalhadores também não compreendiam o Reino de Deus
\pause
\item[•] Os últimos recebem primeiro}
\pause

\begin{itemize}
{\huge \item[•] A estratégica pedagogia de Jesus}
\end{itemize}

\end{itemize}
\end{frame}

\begin{frame}{O Texto da Parábola}
\begin{itemize}
{\huge \item[•] O que Jesus queria ensinar?
\item[•] Primeiro os Judeus, depois os Gentios}
\end{itemize}
\end{frame}

\begin{frame}{Questões Bíblicas}

\begin{itemize}
{\huge\item[•] Na comunidade Cristã, qual deve ser a postura do crente, conforme a orientação de Mateus 20.25-28?}
\end{itemize}
\end{frame}

\begin{frame}{Questões Bíblicas}

\begin{itemize}
{\Huge \item[•] O homem é responsável pela sua salvação?
\pause
 - Ef 2.8-9}
\end{itemize}
\end{frame}
\begin{frame}{A Visão do Senhor da vinha}
\begin{itemize}
{\Huge \item[•] \textbf{Senhor da vinha = Deus}}
\begin{itemize}
{\huge \item[•] Conhece a necessidade do Campo - Mt 9.38
\pause
\item[•] Se esforça para que sejam satisfeitas
\pause
\item[•] Ele vai à praça várias vezes para recrutar
\pause
\item[•] Ele sabe da urgência em colher as uvas}
\end{itemize}
\end{itemize}
\end{frame}

\begin{frame}{A Visão do Senhor da vinha}
\begin{itemize}
{\Huge \item[•] Ele é soberano - I Tm 6.15 e Sl 24.8
\pause
\item[•] A fazenda é sua, a vinha é sua, o resultado é seu, o dinheiro é seu...}
\end{itemize}
\end{frame}

\begin{frame}{A visão do senhor da vinha}
\begin{itemize}
{\Huge \item[•] Ele tem a visão mais ampla das necessidades do Reino.
\pause
\item[•] Soberanamente, Deus escolhe e elege}
\end{itemize}
\end{frame}

\begin{frame}{A Graça do Senhor da vinha}

{\Huge $"$No Reino de Deus os princípios do mérito e da capacidade são colocados de lado para que a graça prevaleça.$"$ - Kistermaker}
\end{frame}

\begin{frame}{A Graça do Senhor da vinha}
\begin{itemize}

{\huge \item[•] A Graça Salvadora não está condicionada aos méritos e às vontades humanas!
\pause
\item[•] A Graça vale mais que a justiça parcial humana.}
\end{itemize}
\end{frame}

\begin{frame}{A Graça do Senhor da vinha}

{\huge O Senhor do Reino aplica a salvação mediante Sua Soberana Graça que, de graça, dá a todos os que, pela vocação recebida, arrependem de seus pecados e os confessa, aceitando a Cristo como único e suficiente Salvador! - Ef 1.18-19}
\end{frame}

\begin{frame}{Os Critérios do Senhor da vinha}
\begin{itemize}

{\Huge \item[•] A Doutrina judáica do mérito
\pause
\item[•] Pergunta do Jovem Rico: "O que \textbf{\textcolor{red}{farei}}?
\pause
\item[•] O pedido da mãe de Tiago e João}
\end{itemize}
\end{frame}

\begin{frame}{Os Critérios do Senhor da vinha}

{\Huge $"$No Reino dos Céus, a bondade de Deus prevalece e se revela àqueles que, somente pela Graça, entraram no Renio.$"$ - Kistermaker}
\end{frame}

\begin{frame}{Os Critérios do Senhor da vinha}
\begin{itemize}

{\huge \item[•] A Coroa da Vida é dada de \textbf{\textcolor{red}{Graça}} não importando quanto tempo se trabalhou na $"$\textit{vinha}$"$.
\pause
\\
\{Paulo = Pedro = Ladrão na Cruz\} = \textcolor{blue}{Mesma Graça}}

\end{itemize}
\end{frame}

\begin{frame}{Conclusão}

{\Huge Não temos nenhuma participação na escolha da nossa Salvação, não se deve a nenhum esforço nosso, mas é uma Obra Graciosa de Deus!}
\end{frame}

\begin{frame}{O Sonho - Stênio Március}

\huge Sonhei que eu tinha morrido,\\
Não lembro direito do quê\\
Me vi frente a um alto e belo portão\\
Com uma placa escrito "Céu"
\end{frame}

\begin{frame}{O Sonho - Stênio Március}

\huge Bati com um certo receio,\\
Um anjo saiu pra atender\\
Me disse "Pois não" eu falei\\ 
"Quero entrar, pois aí é o meu lugar"
\end{frame}

\begin{frame}{O Sonho - Stênio Március}

\huge O anjo me disse "Curioso,\\ 
Eu não acho seu nome em nossos registros"\\
Eu disse "Procure num livro antigo,\\ 
Escrito antes que houvesse mundo,\\
E ali achará com a letra do Rei meu nome em tinta vermelha"
\end{frame}

\begin{frame}{O Sonho - Stênio Március}

\huge Alguém entregou para o anjo\\ 
Registros que eu reconheci\\
Compêndio de todas as leis que\\ 
Eu quebrei e os pecados que cometi
\end{frame}

\begin{frame}{O Sonho - Stênio Március}

\huge O anjo olhava os registros visivelmente assutado\\
E me perguntou "Foi assim que viveu?" \\
Eu então respondi que sim\\
"Então como é que você tem coragem de vir nessa porta bater?"
\end{frame}

\begin{frame}{O Sonho - Stênio Március}

\huge Eu disse "Olhe bem no final dessa lista, \\
Você reconhece esta letra?" \\
E o anjo sorrindo me disse "É verdade, \\
O Rei escreveu \textbf{\textcolor{red}{PERDOADO}}"
\end{frame}

\begin{frame}{O Sonho - Stênio Március}

\huge E ao som dessa bela palavra aquele portão se abriu \\
Então eu entrava cantando um hino \\
Que pena que o sonho acabou \\
Ficaram comigo aquelas palavras \\
"Primeiro eu quero ver meu Salvador"
\end{frame}

\begin{frame}
\centering
\Huge \textbf{\textcolor{black}{FIM}}
\end{frame}

\end{document}